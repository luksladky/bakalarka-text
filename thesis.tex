%%% The main file. It contains definitions of basic parameters and includes all other parts.

%% Settings for single-side (simplex) printing
% Margins: left 40mm, right 25mm, top and bottom 25mm
% (but beware, LaTeX adds 1in implicitly)
\documentclass[12pt,a4paper]{report}
\setlength\textwidth{145mm}
\setlength\textheight{247mm}
\setlength\oddsidemargin{15mm}
\setlength\evensidemargin{15mm}
\setlength\topmargin{0mm}
\setlength\headsep{0mm}
\setlength\headheight{0mm}
% \openright makes the following text appear on a right-hand page
\let\openright=\clearpage

%% Settings for two-sided (duplex) printing
% \documentclass[12pt,a4paper,twoside,openright]{report}
% \setlength\textwidth{145mm}
% \setlength\textheight{247mm}
% \setlength\oddsidemargin{14.2mm}
% \setlength\evensidemargin{0mm}
% \setlength\topmargin{0mm}
% \setlength\headsep{0mm}
% \setlength\headheight{0mm}
% \let\openright=\cleardoublepage

%% Character encoding: usually latin2, cp1250 or utf8:
\usepackage[utf8]{inputenc}

%% Further useful packages (included in most LaTeX distributions)
\usepackage{amsmath}        % extensions for typesetting of math
\usepackage{amsfonts}       % math fonts
\usepackage{mathtools}		% additional math operators
\usepackage{amsthm}         % theorems, definitions, etc.
\usepackage{bbding}         % various symbols (squares, asterisks, scissors, ...)
\usepackage{bm}             % boldface symbols (\bm)
\usepackage{graphicx}       % embedding of pictures
\usepackage{fancyvrb}       % improved verbatim environment
\usepackage{natbib}         % citation style AUTHOR (YEAR), or AUTHOR [NUMBER]
\usepackage[nottoc]{tocbibind} % makes sure that bibliography and the lists
			    % of figures/tables are included in the table
			    % of contents
\usepackage{dcolumn}        % improved alignment of table columns
\usepackage{booktabs}       % improved horizontal lines in tables
\usepackage{paralist}       % improved enumerate and itemize
\usepackage[usenames]{xcolor}  % typesetting in color

\usepackage{epstopdf} %to use .eps files in the document
%%% Basic information on the thesis


\usepackage[ruled,vlined]{algorithm2e}
\usepackage{algpseudocode}
\DontPrintSemicolon

%\usepackage{algorithmic}
%%% Writing algorithms

% Thesis title in English (exactly as in the formal assignment)
\def\ThesisTitle{Key reconstruction from the inner state of RC4}

% Author of the thesis
\def\ThesisAuthor{Lukáš Sladký}

% Year when the thesis is submitted
\def\YearSubmitted{2016}

% Name of the department or institute, where the work was officially assigned
% (according to the Organizational Structure of MFF UK in English,
% or a full name of a department outside MFF)
\def\Department{Department of Algebra}

% Is it a department (katedra), or an institute (ústav)?
\def\DeptType{Department}

% Thesis supervisor: name, surname and titles
\def\Supervisor{Milan Boháček}

% Supervisor's department (again according to Organizational structure of MFF)
\def\SupervisorsDepartment{Department of Algebra}

% Study programme and specialization
\def\StudyProgramme{Mathematics}
\def\StudyBranch{Mathematical Methods of Information Security}

% An optional dedication: you can thank whomever you wish (your supervisor,
% consultant, a person who lent the software, etc.)
\def\Dedication{%
Dedication.
}

% Abstract (recommended length around 80-200 words; this is not a copy of your thesis assignment!)
\def\Abstract{%
Abstract.
}

% 3 to 5 keywords (recommended), each enclosed in curly braces
\def\Keywords{%
{RC4} {cryptoanalysis}
}

%% The hyperref package for clickable links in PDF and also for storing
%% metadata to PDF (including the table of contents).
\usepackage[pdftex,unicode]{hyperref}   % Must follow all other packages
\hypersetup{breaklinks=true}
\hypersetup{pdftitle={\ThesisTitle}}
\hypersetup{pdfauthor={\ThesisAuthor}}
\hypersetup{pdfkeywords=\Keywords}
\hypersetup{urlcolor=blue}

% Definitions of macros (see description inside)
\include{macros}

% Title page and various mandatory informational pages
\begin{document}
%\include{title}

%%% A page with automatically generated table of contents of the bachelor thesis

\tableofcontents

%%% Each chapter is kept in a separate file
\chapter*{Introduction}
\addcontentsline{toc}{chapter}{Introduction}

Despite its age, RC4 is still one of the most widely used stream ciphers in the world. It 




Extract the secret key from given inner state at the end of KSA. Select the best algorithm and implement it.


TODO co budu vyuzivat, jak se to v case vyvijelo, cemu se budu venovat, tohle resili nasledujici, ten a ten umel to a to, nasledujici clanek to vylepsil, jak, mym cilem je co( sice to udelali, ale nedodali zdrojaky, je to tam vagne popsane a tak )
\chapter{The RC4 stream cipher}
\renewcommand{\thealgorithm}{}


\begin{algorithm}
\caption{\textbf{KSA}}
\label{ksa}
\begin{algorithmic}
\State \textit{Initialization:}
\For{$ i = 0,\cdots, N-1$}
	\State $S[i] = i$
\EndFor
\State $j = 0$
\State \textit{Scrambling:}
\For{$ i = 0,\cdots, N-1$}
	\State $j = j + S[i] + K[i\Mod{l}]$
\EndFor
\State \Return{$S$}
\end{algorithmic}	
\end{algorithm}



\begin{algorithm}
\caption{\textbf{PRGA}}
\label{prga}
\begin{algorithmic}
\State \textit{Initialization}
\State $i = 0$
\State $j = 0$
\State \textit{Keystream generation loop}

\State i = i + 1
\State j = j + S[i]
\State Swap(S[i],S[j])
\State t = S[i] + S[j]
\State\Return{$S[t]$}

\end{algorithmic}
\end{algorithm}

\section{Title of the first subchapter of the first chapter}

\section{Title of the second subchapter of the first chapter}

\chapter{Previous attacks}

Two broad approaches - KSA, PRGA

Distinguishing attacks = PRGA-based



Weak keys - KSA


IV mode - WEP


State recovery attacks


RC4 nomore

\chapter{Theoretical analysis of the~KSA}

Hlavni myslenka, nektere klice jsou pravdepodobnejsi nez jine, z toho vnitrniho stavu!! 







\begin{thm}
	Probability of the product of independent events is the product of probabilities, implicitly assumed it is independent. 
\end{thm}


\begin{notation}
	
	\begin{itemize}
		\item 	$ \K{a}{b} \coloneqq \sum\limits_{i=a}^{b}K[i]$
		
		\item 	Let $ j_{i} $ be the pointer $ j $ in i-th round of the KSA. 
	\end{itemize}
	
\end{notation}

\section{Roos bias}
	\begin{lemma}
		TODO prerekvizita vety 1
	\end{lemma}
	
	\begin{lemma}
		TODO prerekvizita vety 1
	\end{lemma}
	
	
	\begin{thm}{\cite{GoMa}}
		Assume that during the KSA the index j takes its values uniformly at random from $ \Z_{N} $. Then $ \forall 0 \leq i \leq r-1, 1 \leq r \leq N $
		
		\[ \Pr(S_{r}[i] = \K{0}{i} + \dfrac{i(i+1)}{2})   \geq (\dfrac{N - i}{N})(\dfrac{N-1}{1})^{\frac{i(i+1)}{2}+r}+\dfrac{1}{N} \]
	\end{thm}
	
	\begin{proof}
		TODO
	\end{proof}
	
	\begin{cor}
		TODO zobecneni na posledni kolo nebo predchozi vetu rovnou smerovat tam?
\end{cor}
	
TODO tabulka s aktualnimi hodnotami 

TODO to same pro InvS


\begin{thm}
	After the complete KSA, 
	\[  \Pr(S_{N}[S_{N}[y]] = \K{0}{i} + \dfrac{i(i+1)}{2}) \approx  \]
	
	clanek 4, appendix, na zacatku graf
\end{thm}

TODO - tabulka s aktualnimi hodnotami, dukaz




TODO zobecneni na sekvence


TODO inverzni sekvence

TODO vyyiti tohoto na ziskani klice - rovnice


\section{Substracting equations}

Let $i_{1} < i_{2} $. If $ C_{i_{1}} = \K{0}{i_{1}} $ and $ C_{i_{2}} = \K{0}{i_{2}} $, then we can substract the values and get
\[ C_{i_{2}} - C_{i_{1}} = \K{0}{i_{2}} - \K{0}{i_{1}} = \K{i_{1} + 1}{i_{2}}	\].

This holds with the product of the individual probabilities of $ C_{i} $


\section{Useful distributions for The Key Recovering Algorithm}

\begin{defn}
If $ j_{i} = S[i] $, we call this as event 1 has occured for index $ i $, and denote as $ E_{1} $.
\end{defn}

\begin{thm}
\[	P(S[i] = j_{i}) \geq (1-\dfrac{1}{N})^{i}(1-\dfrac{i-1}{N})(1-\dfrac{1}{N})^{N-i-1}+ \dfrac{1}{N} \]
\end{thm}

\begin{defn}
	If $ j_{i} = S[i] $, we call this as event 1' has occured for index $ i $, and denote as $ E'_{1} $.
\end{defn}

\begin{thm}
	\[	P(S^{-1}[i] = j_{i}) \geq (\dfrac{i}{N})(1-\dfrac{1}{N})^{N-1}+ \dfrac{1}{N}\]
\end{thm}


\begin{proof}
		$ (1-\dfrac{1}{N})^{i}(\dfrac{i}{N})(1-\dfrac{1}{N})^{N-i-1}+ \dfrac{1}{N} $
\end{proof}

\begin{thm}
	$ P(S[i] = j_{i} \vee S^{-1}[i] = j_{i}) $ used for BuildKeyTable algorithm in paper 1
\end{thm}


TODO $ P(S[S[i]] = j_{i}) $ - dukazy nikde nejsou...

TODO $ P(S^{-1}[S^{-1}[i]] = j_{i}) $

TODO $ P(S[S[S[i]] = j_{i}) $




\chapter{Going back to permutation after KSA}

\begin{itemize}
	\item I have state, $ i,j $, number of rounds. 
	\item I have state, $ i,j $, not number of rounds. 
	\item I have state, number of rounds. 
\end{itemize}
\chapter{The Key Recovering Algorithm}

Tady bych chtel popsat ten, ktery ve finale pouziju. Nebo vsechny?


\TODO{Some keys are more probable than others, we will sort them according to weight and try first $ nc $ number of candidates? Dat to tam?}\label{key}

\include{epilog}

%%% Bibliography
\include{bibliography}

%%% Figures used in the thesis (consider if this is needed)
%\listoffigures

%%% Tables used in the thesis (consider if this is needed)
%%% In mathematical theses, it could be better to move the list of tables to the beginning of the thesis.
%\listoftables

%%% Abbreviations used in the thesis, if any, including their explanation
%%% In mathematical theses, it could be better to move the list of abbreviations to the beginning of the thesis.
%\chapwithtoc{List of Abbreviations}

%%% Attachments to the bachelor thesis, if any. Each attachment must be
%%% referred to at least once from the text of the thesis. Attachments
%%% are numbered.
%%%
%%% The printed version should preferably contain attachments, which can be
%%% read (additional tables and charts, supplementary text, examples of
%%% program output, etc.). The electronic version is more suited for attachments
%%% which will likely be used in an electronic form rather than read (program
%%% source code, data files, interactive charts, etc.). Electronic attachments
%%% should be uploaded to SIS and optionally also included in the thesis on a~CD/DVD.
%\chapwithtoc{Attachments}

\openright
\end{document}
